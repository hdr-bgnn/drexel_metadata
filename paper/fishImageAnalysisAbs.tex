

\subsection{Fish Image Analysis}

Image analysis has been utilized to examine and process images of fish
for well over two decades \cite{Zion2000InvivoFS}. It is an important
application of technology both for marine science, in the study of aquatic
habitats and ecosystems, and for food the industry, in the development
of automated fish sorting and grading systems, as well as fisheries
management. The main focus of
these computational analyses has been the recognition and classification 
of the fish present in an image.

\textbf{Fish Recognition and Classification}\newline
Alsmadi 2010 \cite{Alsmadi2010FishRB},
 our aim was to recognize an isolated pattern of interest in the image based on the combination between robust features extraction. Where depend on size and shape measurements, that were extracted by measuring the distance and geometrical measurements.

Alsmadi 2019 \cite{Alsmadi2019RFE},
Based on feature selection and combination theory between major extracted features, this study attempts to establish a system that could recognize fish object within the image utilizing texture, anchor points, and statistical measurements. Then, a generic fish classification is executed with the application of an innovative classification evaluation through a meta-heuristic algorithm known as Memetic Algorithm (Genetic Algorithm with Simulated Annealing) with back-propagation algorithm (MA-B Classifier).

Hasija 2017 \cite{Hasija2017FishSC},
This paper proposes a novel method based on an improved image-set matching approach, which makes use of Graph-Embedding Discriminant Analysis. In contrast to the state-of-the-art methods, which operate on single input images, our method makes, use of explicit image set matching which renders it robust computationally.

Hu 2012 \cite{HuJing2012Fscb},
This paper presents a novel method of classifying species of fish based on color and texture features and using a multi-class support vector machine (MSVM). Fish images were acquired and sent by smartphone

Iqbal 2021 \cite{Iqbal2021AutomaticFS},
In this paper, we presented an automated system for identification and classification of fish species. It helps the marine biologists to have greater understanding of the fish species and their habitats. The proposed model is based on deep convolutional neural networks. It uses a reduced version of AlexNet model comprises of four convolutional layers and two fully connected layers. A comparison is presented against the other deep learning models such as AlexNet and VGGNet. The four parameters are considered that is number of convolu- tional layers and number of fully-connected layers, number of iterations to achieve 100% accuracy on training data, batch size and dropout layer. The results show that the proposed and modified AlexNet model with less number of layers has achieved the testing accuracy of 90.48% while the original AlexNet model achieved 86.65% over the untrained bench- mark fish dataset. The inclusion of dropout layer has enhanced the overall performance of our proposed model. It contain less training images, less memory and it is also less compu- tational complex.

Li 2014 \cite{Li2014IdentificationOF}
Abstract - For more fish varieties, to the subsequent processing and marketing, which is necessary to classify the types of fish.This study is based on image processing technology to classify the types of fish, four fish to make use of the existing image acquisition device for collecting samples, through the MATLAB software for image preprocessing, such as fish gray, binarization, image enhancement, contour extraction to extract the 11 feature parameters of four fish species, such as using the principal component analysis (PCA) to 11 characteristic parameters for dimension reduction, this study took four principal component.Then use SPSS software to establish fisher and mahalanobis distance model, the combination of four principal component reuse component to build a model to classify the four different kinds of fish.Through SPSS software simulation and identification results show that the average recognition rate of 96.67%, which can be well applied to the fish species identification technology.


Banan 2020 \cite{Banan2020DeepLA}
Hence, in this study, a deep learning neural network as a smart, real-time and non-destructive method was developed and applied to automate the identification of four economically important carp species namely common carp (Cyprinus carpio), grass carp (Ctenopharingodon idella), bighead carp (Hypophtalmichthys nobilis) and silver carp (Hypophthalmichthys molitrix). The obtained results proved that our approach, evaluated through 5-fold cross-validation, achieved the highest possible accuracy of 100 %. The achieved high level of classification accuracy was due to the ability of the suggested deep model to build a hierarchy of self-learned features, which was in accordance with the hierarchy of these fish’s identification keys. In conclusion, the proposed convolutional neural network (CNN)-based method has a single and generic trained architecture with promising performance for fish species identification.

Ogunlana 2015 \cite{Ogunlana2015FCU},
In this paper, a Support Vector Machine (SVM)-based technique for the elimination of the limitations of some existing techniques and improved classification of fish species is proposed. The technique is based on the shape features of fish that was divided into two subsets with the first comprising 76 fish as training set while the second comprises of 74 fish as testing set. The body and the five fin lengths; namely anal, caudal, dorsal, pelvic and pectoral were extracted in centimeter (cm). Results based on the new technique show a classification accuracy of 78.59%, which is significantly higher than what obtained for ANN, KNN and K-mean clustering-based algorithms.

Rodrigues 2015 \cite{RodriguesMarcoT.A2015Ecda},
This paper proposes five different schemes for automatic classification of fish species. These schemes make the species recognition based on image sample ana- lysis. Different techniques have been combined for build- ing the classifiers: three feature extraction techniques (PCA, SIFT and SIFT ? VLAD ? PCA), three data clustering algorithms (aiNet, ARIA and k-means) and three input classifiers (k-NN, SIFT class. and k-means class) are tested. When compared to common methodologies, which are based on human observation, it is believed that these schemes are able to provide significant improvement in time and financial resources spent in classification. Two datasets have been considered: (1) a dataset with image samples of six fish species which are perfectly conserved in formaldehyde solution, and; (2) a dataset composed of images of four fish species in real-world conditions (in vivo). The five proposed schemes have been evaluated
in both datasets, and a ranking for the methods has been derived for each one.

Saitoh 2015 \cite{Saitoh2015ImagebasedFR},
We are studying image-based fish identification. Most of traditional approaches used a fish image which was easy to extract a fish region with a white background or uniform background for automatic processing. This research adapted an approach to give several points by manual operation by the user. The proposed approach is able to accept the fish image in the complicated background taken on the rocky place. Furthermore, to investigate the efficient features for fish recognition, we defined various features, such as, shape features, local features, and six kinds of texture features. We collected 129 species under various photography conditions, and the proposed method was carried out to it. As the results, it was confirmed that a combination features with geometric features and BoVW models obtained the highest recognition accuracy.

Salman 2016 \cite{Salman2016FishSC},
Underwater video and digital still cameras are rapidly being adopted by marine scientists and managers as a tool for non-destructively quantifying and measuring the relative abundance, cover and size of marine fauna and flora. Imagery recorded of fish can be time consuming and costly to process and analyze manually. For this reason, there is great interest in automatic classification, counting, and measurement of fish. Uncon- strained underwater scenes are highly variable due to changes in light intensity, changes in fish orientation due to movement, a variety of background habitats which sometimes also move, and most importantly simi- larity in shape and patterns among fish of different species. This poses a great challenge for image/video processing techniques to accurately differentiate between classes or species of fish to perform automatic clas- sification. We present a machine learning approach, which is suitable for solving this challenge. We demon- strate the use of a convolution neural network model in a hierarchical feature combination setup to learn species-dependent visual features of fish that are unique, yet abstract and robust against environmental and intra-and inter-species variability. This approach avoids the need for explicitly extracting features from raw images of the fish using several fragmented image processing techniques. As a result, we achieve a single and generic trained architecture with favorable performance even for sample images of fish species that have not been used in training. Using the LifeCLEF14 and LifeCLEF15 benchmark fish datasets, we have demonstrated results with a correct classification rate of more than 90%.

Sayed 2018 \cite{SayedGehadIsmail2018AAFS},
This paper proposed an automated fish species identification system based on a modified crow search optimization algorithm. Median filtering is applied for image smoothing and removing noise through reducing the variation of intensities between the neighbors. Then, a k-mean clustering algorithm is used to segment the fish image into multiple segments. Shape-based and texture-based feature extraction process for classification is presented. A new modified binary version of crow search algorithm is proposed to reduce the data dimensionality of the extracted features. Finally, support vector machine and decision trees are implemented for classification and the fish species are classified based on either their class including Actinopterygii and Chondrichthyes or based on their order. Total of 270 images with different species, classes and orders are used for evaluation of the proposed system. The experimental results show that the proposed system achieves the highest classification accuracy compared to state-of-the-art algorithms. Also, the results show that the overall fish species identification system obtains on average of 10 folds, 96% classification accuracy for classification based on class and 74% for classification based on fish order.

Sharmin 2019 \cite{Sharmin2019MVB},
the new generation people of Bangladesh lacks the knowledge of local freshwater fish. For this problem, a solution has been found with the collaboration of vision-based technology. As a solution, a machine-vision based local freshwater fish recognition system is presented that can be proceed with an image of fish captured with a mobile or handheld device and recognize the fish in order to introduce the fish. To demonstrate the utility of the proposed expert system, several experiments are performed. At first, a set of fourteen features, which consists of four types of features, are presented. Then the color image has been converted into gray-scale image and the gray-scale histogram is formed. Image segmentation takes place using histogram-based method and then the features are extracted. PCA is used for decreasing the feature numbers. Three classifiers are used for recognizing fish, where SVM gives the highest accuracy showing a value of 94.2%.

Sung 2017 \cite{Sung2017VisionBR},
Underwater vision has specific characteristics such as high attenuation of lights, severe noise and haze in the images. For real-time fish detection using underwater vision, this paper proposes convolutional neural network based techniques based on You Only Look Once algorithm. Actual fish video images were used to evaluate the reliability and accuracy of the proposed method. As a result, the network recorded 93% classification accuracy, 0.634 intersection over union between predicted bounding box and ground truth, and 16.7 frames per second of fish detection. It also outperforms another fish detector using sliding window algorithm and classifier trained with histogram of oriented gradient features and support vector machine.

Xu 2021 \cite{Xu2021TransferLA},
Scientific studies on species identification in fish have considerable significance in aquatic ecosystems and quality evaluation. The morphological differences between different fish species are obvious. Machine learning methods use artificial prior knowledge to extract fish features, which is time-consuming, laborious, and subjective. Recently, deep learning-based identification of fish species has been widely used. However, fish species identification still faces many challenges due to the small scale of fish samples and the imbalance of the number of categories. For example, the model is prone to being overfitted, and the performance of the classifier is biased to the fish species of most samples. To solve the above problems, this paper proposes a fish species identification approach based on SE-ResNet152 and class-balanced focal loss. First, visualization analysis and image preprocessing of fish datasets are carried out. Second, the SE-ResNet152 model is constructed as a generalized feature extractor and is migrated to the target dataset. Finally, we apply the class-balanced focal loss function to train the SE-ResNet152 model, and realize fish species identification on three fish image views (body, head, and scale). The proposed method was tested on the Fish-Pak public dataset and achieved 98.80%, 96.67%, and 91.25% accuracy on the three fish image views, respectively. To ensure the superior performance of the proposed method, we performed an experimental comparison with other methods involving SENet154, DenseNet121, ResNet18, ResNet152, VGG16, cross-entropy, and focal loss. Comprehensive empirical analyses reveal that the proposed method achieves good performance on the three fish image views and outperforms common methods.

Zion 2000 \cite{Zion2000InvivoFS},
An image-processing algorithm, applied to images of common carp (Cyprinus carpio), St. Peter’s fish (Oreochromis sp.) and grey mullet (Mugil cephalus), successfully discriminated among the species. Fish images were acquired while they were swimming in an aquarium with their side to the camera. The algorithm was based on the method of moment-invariants (MI) coupled with geometrical considerations and was, therefore, insensitive to fish size, two-dimensional orientation and location in the camera’s field of view. One hundred and forty three images (47 grey mullet, 43 St. Peter’s fish and 53 carp images) were acquired and divided into two sets: 20 grey mullet, 20 St. Peter’s fish and 20 carp images in one set and the rest of the images in the other set. Each of these two sets was used as a training set for selection of feature thresholds, which were then applied to the other set as a test case (two-fold cross-validation test). Fish species identification reached 100, 91 and 91% for grey mullet, carp and St. Peter’s fish, respectively. To the best of our knowledge this is the first report on successful discrimination among fish species in vivo.

\textbf{Orientation, Length, Size, Weight and Shape}\newline
Azarmdel 2019 \cite{Azarmdel2019DevelopingAO},
Fish processing in small and medium fish supplying centers requires an intelligent system to operate on different sizes. Therefore, an image processing algorithm was developed to extract the proper head and belly cutting points according to the trout dimensions. The algorithm detects the fish orientation and location of pectoral, anal, pelvic, and caudal fins. In this study, each of the trout images was divided into slices along its length in order to segment the fins and extract cutting points. The channel ‘B’ of RGB color space was considered in both initial segmentation and fin detection stages among the examined channels of RGB, HSV, and L*a*b* color spaces. The back-belly and head-tail sides were detected with an accuracy of 100% based on gray intensity values and head to tail ratio, respectively. Furthermore, performing an analysis of variance (ANOVA) resulted in an F-value of 64.82 among the fins. Conducting a t-test among the mean intensity values of the fins and non-fin regions of channel ‘B’ resulted in the highest distinction with t-values of 90.30, 78.07, 74.28, and 86.01 with p < 0.01 for the pectoral, pelvic, anal, and caudal fins paired with the corresponding non-fin region, respectively. The results showed that the selected ‘B’ channel is the adequate one for fin segmentation. The fin detection process showed an overall sensitivity, specificity, and accuracy of 86.05%, 99.97%, and 99.87%, respectively. By solving the line determination error in 8.24% and the extra object error in 4.12% of the samples, the overall fin identification accuracy was 100%. Finally, after extracting the fin regions, the start point of the pectoral fin and the end point of the anal fin will be applied in the trout processing system as the head and belly cutting points, respectively.

Balaban 2010 \cite{Balaban2010UsingIA},
After harvesting , salmon is sorted by species, size, and quality. This is generally manually done by op-erators. Automation would bring repeatability, objectivity, and record-keeping capabilities to these tasks. Machinevision (MV) and image analysis have been used in sorting many agricultural products. Four salmon species weretested: pink (Oncorhynchus gorbuscha), red (Oncorhynchus nerka), silver (Oncorhynchus kisutch), and chum (On-corhynchus keta). A total of 60 whole fish from each species were first weighed, then placed in a light box to taketheir picture. Weight compared with view area as well as length and width correlations were developed. In additionthe effect of “hump” development (see text) of pink salmon on this correlation was investigated. It was possible topredict the weight of a salmon by view area, regardless of species, and regardless of the development of a humpfor pinks. Within pink salmon there was a small but insignificant difference between predictive equations for theweight of “regular” fish and “humpy” fish. Machine vision can accurately predict the weight of whole salmon forsorting.

Hao 2015 \cite{Hao2015TheMO}, A REVIEW
This study reviewed the methods of fish size measurement through machine vision. Length and area are important information that can help fishers manage fish scientifically and conveniently. This information could be used to calculate the volume and weight according to their relation; other information could also be calculated. Machine vision is more effective, economical and faster than traditional methods.

Konovalov 2017 \cite{KonovalovD2017RDfA},
Fast and low cost image collection and processing is often required in aquaculture farms for quality/size attributes and breeding programs. For example, the absolute physical dimensions of fish (in millimeters or inches) could be estimated from electronic images. The absolute scale of the photographed fish is often unknown or requires additional hardware, data- collection and/or management overheads. One cost and time effective solution is to capture the absolute scale (in pixels-per- millimeter or dots-per-inch) by including a measuring ruler in the photographed scene. To assist that type of workflow, we designed a relatively simple image-processing algorithm that automatically located a sufficiently large section of the ruler in a given image. The algorithm utilized the Fast Fourier Transform and was designed to be free from adjustable parameters and therefore did not require training or calibration. The algorithm was tested on 445 images of Barramundi (Asian sea bass, Lates calcarifer), where a millimeter-graded ruler was included in each image. The algorithm achieved precision of 98% (on the original, 10, 20, 70, 80 90 degree rotated images) and 95-96% on 40, 50, 60 degree rotated images. \newline
Konovalov 2018 \cite{Konovalov2018AutomaticSO},
In aquaculture breeding programs where large numbers of fish need to be rapidly phenotyped, the absolute physical dimensions of fish (in millimeters or inches) are often required to be extracted from electronic images in order to measure the size of the fish. While it is possible to infer the length of the fish in pixels, the absolute scale of the image (in pixels-per-millimeter or dots-per- inch) is largely unknown without a reference grid, or requires additional hardware, data collection and/or record-keeping management overheads. One cost and time effective solution is to capture the absolute scale by including a measuring ruler in the photographed scene and from which a computer program can automatically identify the scale of the photo and calculate fish morphometric measurements. To assist such workflow, this study developed an algorithm that automatically detects a ruler in a given image, and automatically extracts its scale as distance (in fractional number of pixels) betw􏱜􏱜􏰬 􏰦􏱬􏱜 􏰪􏰣􏰴􏱜􏰪􏰱􏰤 􏱼􏰪􏲂􏱙􏰣􏲂􏰦􏲋􏱟􏰬 􏰨􏲂􏰪􏲀􏰤. The algorithm was applied to 445 publicly available images of barramundi or Asian seabass (Lates calcarifer), where a millimeter-graded ruler was included in each image. Convolutional Neural Network (CNN) was trained to segment the images into ruler, background, fish and label sections. Then the distance-extraction algorithm was applied to the ruler section of the images. The false-negative rate was less than 2%, where the ruler graduation distances could not be extracted in only 2-6 (out of 445) images even when the test images were rotated up to 90 degrees. The mean absolute relative error (MARE) of the inferred distances was 1-2%.\newline
Konovalov 2019 \cite{Konovalov2019AutomaticWE},
Approximately 2,500 weights and corresponding images of harvested Lates calcarifer (Asian seabass or barra- mundi) were collected at three different locations in Queensland, Australia. Two instances of the LinkNet-34 segmentation Convo- lutional Neural Network (CNN) were trained. The first one was trained on 200 manually segmented fish masks with excluded fins and tails. The second was trained on 100 whole-fish masks. The two CNNs were applied to the rest of the images and yielded automatically segmented masks. The one-factor and two-factor simple mathematical weight-from-area models were fitted on 1072 area-weight pairs from the first two locations, where area values were extracted from the automatically segmented masks. When applied to 1,400 test images (from the third location), the one- factor whole-fish mask model achieved the best mean absolute percentage error (MAPE), MAPE = 4.36%. Direct weight-from- image regression CNNs were also trained, where the no-fins based CNN performed best on the test images with MAPE = 4.28%.


\textbf{Fish Quality}\newline
Lalabadi 2020 \cite{Lalabadi2020FishFC},
Developing new techniques to determine fish freshness and quality can enhance nutritional value of the overall household food basket. In this research, digital image analysis was utilized to assess the freshness of rainbow trout fish by tracing the color attributes of its eyes and gills. The image data were collected from left and right eyes and gills in a 10-day ice-storage duration, and color components were extracted in RGB, HSV, and L*a*b* color spaces. Analysis of variance revealed that the RGB components of both eyes and gills had a significant change towards getting brighter during the ice-storage. Feature extraction was fulfilled from the color spaces, and then artificial neural networks (ANNs) and support vector machines (SVMs) were applied for classification of the ice-storage durations. The overall accuracies of the developed models demonstrated that the ANN somewhat outperformed the SVM for both the extracted features from the eyes and gills. Moreover, the gills’ features could describe the variance in the storage durations more efficiently than those extracted from the eyes. Finally, it was concluded that the applied colorimetric system along with the developing models could be employed as a successful non-destructive approach for evaluation of fish freshness.

Taheri-Garavand 2019 \cite{TaheriGaravand2019RealtimeNM},
In the current research, the potential of a novel method based on the artificial neural network was investigated to diagnose the freshness of common carp (Cyprinus carpio) during ice storage. Fish as an aquaculture product has high nutrients and low-fat content. So, people have consumed it as a safe and high-value foodstuff in their daily diet. Investigation of fish freshness is proposed as a significant issue in the aquaculture industry since fish spoils rapidly. The applied system of this study is comprised of the following steps: First, images of samples were captured and the pre-processing operation was done on the images. Then, particular channels including R, G, B, H, S, I, L*, a*, and b* were computed. Next, feature extraction was performed to obtain 6 types of texture features from each channel. Afterward, the hybrid Artificial Bee Colony-Artificial Neural Network (ABC-ANN) algorithm was applied to select the best features. Finally, the Support Vector Machine (SVM), K-Nearest Neighbor (K-NN) and Artificial Neural Network (ANN) algorithms as the most common methods were used to classify fish images. The best performance of the K-NN classifier was calculated in the k = 8 neighborhood size with the accuracy of 90.48. The best kernel function for the SVM algorithm was polynomial with C, sigma, and accuracy of 1, 2 and 91.52 percent, respectively. In this system, the input layer has consisted of 22 neurons based on the feature selection operation and 4 classes including most fresh, fresh, fairly fresh and spoiled have been used as the number of output layer. At the end, the best results of the MLP networks were achieved by LM learning algorithm and 6 neurons in the hidden layer with the 22–10–4 topology and accuracy of 93.01 percent. The achieved results demonstrate the high performance of the ANN classifier for evaluation of common carp freshness during ice storage as a rapid, accurate, non-destructive, real-time and automated method. It shows the potential of computer vision method in combination with artificial neural networks as an intelligent technique for evaluation of fish freshness.

Tappi 2017 \cite{Tappi2017ComputerVS},
The evaluation of fish freshness can be per- formed using chemical, sensory and physical methods. Besides sensory methods, several instrumental techniques have been applied with the objective of replacing sensory assessment. The aim of this study was to set up and test objective physical methods mainly based on computer vision system (CVS) to assess red mullet (Mullus barba- tus) freshness evolution during 10 days of storage, at two different storage temperatures (0 and 4 °C). To check the effectiveness of the purposed physical methods, CVS fea- tures (loss in the epidermis pigmentation, development of gill mucus and eye concavity index) and firmness have been compared with chemical trimethylamine content and sensory (QIM) attribute scores. As expected, fish degrada- tion was faster at the higher temperature. Instrumental tex- ture evaluation of fish by penetration test enabled to detect distinctive firmness changes due to onset and resolution of rigor mortis, and the successive tenderization phenomenon. Among CVS parameters, the epidermis pigmentation loss, and particularly the eye shape modification (eye concavity index) evidenced a high sensibility for the estimation of fresh red mullet quality loss, as a function of the two differ- ent storage conditions, and a good agreement with trimeth- ylamine content and QIM response evolution.

\textbf{Computer Vision Overviews Related to Fish}\newline
Saberioon 2017 \cite{Saberioon2017ApplicationOM}, A decent, somewhat 
recent overview.


\textbf{Unbending a Fish}\newline
Mu{\~n}oz-Benavent 2018 \cite{MuozBenavent2018EnhancedFB},
Williams 2020 \cite{Williams:2020:UIF},

\textbf{Not Used}\newline
(Kinda vague. More about food, then fish, G{\"u}m{\"u}ş 2011 \cite{Gm2011MachineVA}),
(In Turkish, Iscimen 2014 \cite{Iscimen2014ImageAM},
Iscimen 2015 \cite{Iscimen2015ClassificationOF}),
(Not based on images (sonar?) Kinjo 2014 \cite{KinjoAtsushi2014Scoi}),
(Too basic, Wang 2015 \cite{Huihui2015StudyOT}),
(Not so interesting, a very engineered, constrained imaging set up, Miranda 2017 \cite{MIRANDA201741}),

(A review of all types of sensing systems for quality checkin, Bernardo 2020 \cite{Bernardo2020FishQI}),
(Another review, but in Trends in Analytical Chemistry. I think we can
skip it. To so technical, Dowlati 2012 \cite{Dowlati2012ApplicationOM}),

(Simple industrial system. Computing length, Sung 2020 \cite{Sung2020AutomaticGF}),
(It's 3D. Don't want to go there, Bock 2018 \cite{BockAlexander2018TITE}),

(A review that is not quite focused enough, Petrov 2020 \cite{Petrov2020Overview}),
(Don't have access to it, and I have a newer review, Zion 2012 \cite{Zion2012ReviewTU}),


